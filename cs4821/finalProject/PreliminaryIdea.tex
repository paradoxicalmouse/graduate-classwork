\documentclass[12pt,letterpaper]{report}
\usepackage[utf8]{inputenc}
\usepackage{amsmath}
\usepackage{amsfonts}
\usepackage{amssymb}
\usepackage[left=1in,right=1in,top=1in,bottom=1in]{geometry}
\author{Taylor B. Morris}
\title{Preliminary Idea: Game Sales}
\begin{document}
\maketitle

\section{Topic Area}
I would like to explore various facets of the game sales dataset to search for patterns in known game stats such as Genre, Reviewer Ratings, Company, and System to discover any commonality in games which have higher or low sales. 

\section{Problem and Goal}
Games take thousands of dollars in manpower and resources to develop, and it would be an essential asset to game companies to predict how well a game will fair in the sales market before they move on to sales. By looking at commonalities between different games sold over the years, I believe a clear prediction can be made about how well a game will sell in a given market. Additionally, knowing the factors which make a game more likely to sell well would be important to game companies, as this information could cause them to create games closer to what the current audience is expecting. Finally, knowing which market a game might get the best user ratings in would help for the manufactures to decide where to focus sales.

\section{Problem Formation}
I plan to use regression techniques to predict game sales, and classification techniques to predict user ratings on a game.

\section{Literature Search}
I plan to search the IEEE journal, AAAI conferences, and google scholar database of journals.

Possible background ideas:


http://ieeexplore.ieee.org/abstract/document/5677530/


https://vvvvw.aaai.org/Papers/Symposia/Spring/2006/SS-06-03/SS06-03-030.pdf

\section{Data Resources}
I am going to use a dataset found on kaggle.com called "Video Game Sales with Ratings". It is a relatively large dataset, with every game released  before December 2016. There are 16 features, with older games missing 6 of the features. There are around 6,900 complete cases.

\section{Evaluation}
For evaluation, I plan to use SSE and r statistics to evaluate accuracy.

\end{document}