\documentclass[12pt,letterpaper]{report}
\usepackage[utf8]{inputenc}
\usepackage{enumerate}
\usepackage{amsmath}
\usepackage{amsfonts}
\usepackage{amssymb}
\usepackage[left=1in,right=1in,top=1in,bottom=1in]{geometry}
\author{Taylor B. Morris}
\title{OS File Systems Homework}
\begin{document}
\makeatletter
{\huge\noindent\@title\large\\\@author\\\@date}
\makeatother
\begin{enumerate}

%Question 1
\item One benefit of contiguous allocation is that read requests are significantly 
faster for requests of sequential blocks as compared to requests of non-sequential 
blocks. Additional benefits is that it would be more memory efficient to keep track of 
all the blocks of a file over separated blocks, as the system then only has to track
one range per file, instead of several. Contiguous allocation is rarely use because
files are commonly deleted and resized, meaning that a contiguous allocation would end
up wasting space - you would need a contiguous block of exactly the right size in order
for the contiguous allocation to work and not be wasting space - i.e. say you have 200
blocks of memory free in a row, but have 190 blocks of information to store there. Now,
you only have 10 blocks of memory free in that location. As that free value gets smaller,
nothing will be able to fill the free space. Repeated several times throughout the drive,
you could end up with hundreds of blocks of wasted memory. Additionally, contiguous
allocation requires there to actually be contiguous blocks, making it difficult to store
information as the drive gets full.

%Question 2
\item File Types: 
\begin{enumerate}[{(Type }1{)}]
\item Regular file: 
\item Directory: 
\item Character device: 
\item Block device:
\item Named pipe:
\item Socket:
\item Symbolic link:
\end{enumerate}

%Question 3
\item

%Question 4
\item

%Question 5
\item

%Question 6
\item

%Question 7
\item

%Question 8
\item



\end{enumerate}
\end{document}
