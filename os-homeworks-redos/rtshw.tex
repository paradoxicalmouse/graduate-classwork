\documentclass[12pt,letterpaper]{report}
\usepackage[utf8]{inputenc}
\usepackage{amsmath}
\usepackage{amsfonts}
\usepackage{amssymb}
\usepackage[left=1in,right=1in,top=1in,bottom=1in]{geometry}
\author{Taylor B. Morris}
\title{OS RTS Homework Re-Do}
\begin{document}
\makeatletter
{\huge\noindent\@title\large\\\@author\\\@date}
\makeatother
\begin{enumerate}

%Question 1
\item
\begin{enumerate}
\item $T_{minor}$ is the greatest common divisor between the periods, $3$.
\item $T_{major}$ is the least common multiple between the periods, $12$.
\item There are $T_{major}/T_{minor}=4$ slots in the major cycle.
\item For $\tau_1$, only $1$ slot may pass after time $0$ before it is scheduled again. For $\tau_2$, $2$ slots may pass after time $0$ before it is scheduled again. Finally, $4$ slots may pass after time $0$ before $\tau_3$ is scheduled again.
\item Yes, a feasible schedule exists. Slot bounds indicated by a horizontal line.\\
\begin{tabular}{|c|c|}
\hline
Time Unit & Task \\
\hline
0-0 & $\tau_1$ \\
1-2 & $\tau_2$ \\
\hline
3-3 & $\tau_1$ \\
4-5 & $\tau_3$ \\
\hline 
6-6 & $\tau_1$ \\
7-8 & $\tau_2$ \\
\hline
9-9 & $\tau_1$ \\
10-11 & \\
\hline
\end{tabular}
\item Function in table form below.\\
\begin{tabular}{|c||c|c|c|}
\hline
Time & $\tau_1$ & $\tau_2$ & $\tau_3$ \\
\hline\hline
0 & 1 & 0 & 0\\
1 & 0 & 1 & 0\\
2 & 0 & 1 & 0\\
3 & 1 & 0 & 0\\
4 & 0 & 0 & 1\\
5 & 0 & 0 & 1\\
6 & 1 & 0 & 0\\
7 & 0 & 1 & 0\\
8 & 0 & 1 & 0\\
9 & 1 & 0 & 0\\
10 & 0 & 0 & 0\\
11 & 0 & 0 & 0\\
\hline
\end{tabular}
\end{enumerate}

\item \begin{enumerate}
\item $\tau_2$ should have the highest priority, assuming RM or DM, as it has the smallest period.
\item $\tau_1$ should have the lowest priority, assuming RM or DM, as it has the largest period.
\item The load is $\sum_i C_i/\tau_i=2/8 + 2/6= 7/12$.
\item Yes, they have a feasible static priority assignment. Assigning priorities with $\tau_2$ having the highest priority and $\tau_1$ having the lowest priority will obviously give a static priority assignment, both completing in 4 time blocks total, well before their deadline of the next scheduling.
\item Adding in $\tau_3$ will still have a static priority assignment which works, since making $\tau_3$ highest, followed by $\tau_2$, followed by $\tau_1$ will work.
\end{enumerate}
\item
\begin{enumerate}
\item The load is $1/4 + 3/6 + 2/8=1$. Since the load is (less than or) equal to 1, the task has a feasible EDF schedule.
\item EDF schedule shown below:\\
\begin{tabular}{|c|c|}
\hline 
Time & Task\\
\hline
0 & $\tau_1$\\
1 & $\tau_2$\\
2 & $\tau_2$\\
3 & $\tau_2$\\
4 & $\tau_1$\\
5 & $\tau_3$\\
6 & $\tau_3$\\
7 & $\tau_2$\\
8 & $\tau_1$\\
9 & $\tau_2$\\
10 & $\tau_2$\\
11 & $\tau_3$\\
12 & $\tau_1$\\
13 & $\tau_3$\\
14 & $\tau_2$\\
15 & $\tau_2$\\
16 & $\tau_2$\\
17 & $\tau_1$\\
18 & $\tau_3$\\
19 & $\tau_3$\\
20 & $\tau_1$\\
21 & $\tau_2$\\
22 & $\tau_2$\\
23 & $\tau_2$\\
\hline
\end{tabular}
\end{enumerate}
\item \begin{enumerate}
\item The priority of each process under Deadline Monotonic scheduling is its deadline, where lower deadlines (closer deadlines) have higher priority. Let 1 be top priority, and so on. So, $\tau_1$ would have priority 2, $\tau_2$ priority 1, and $\tau_3$ priority 3. 
\item Let $R_i$ be the response time of the $i^{th}$ task. Then, $R_1=3$, $R_2=2$, and $R_3=10$
\item No, a feasible deadline monotonic schedule doesn't exist, as $R_3 > D_3$.
\end{enumerate}
\item For this problem, I will assume P and V are instantaneous and atomic. 
\begin{enumerate}
\item The finish time for $\tau_1$ will be at the 17 second mark, $\tau_2$ at the 9 second mark, and $\tau_3$ at the 18 second mark.
\item The finish times will be $\tau_1$ at the 15 second mark, $\tau_2$ at the 17 second mark, and $\tau_3$ at the 18 second mark. 
\item The finish times will be $\tau_1$ at the 12 second mark, $\tau_2$ at the 16 second mark, and $\tau_3$ at the 18 second mark. 
\end{enumerate}
\end{enumerate}
\end{document}
